% Options for packages loaded elsewhere
\PassOptionsToPackage{unicode}{hyperref}
\PassOptionsToPackage{hyphens}{url}
%
\documentclass[
]{article}
\usepackage{amsmath,amssymb}
\usepackage{iftex}
\ifPDFTeX
  \usepackage[T1]{fontenc}
  \usepackage[utf8]{inputenc}
  \usepackage{textcomp} % provide euro and other symbols
\else % if luatex or xetex
  \usepackage{unicode-math} % this also loads fontspec
  \defaultfontfeatures{Scale=MatchLowercase}
  \defaultfontfeatures[\rmfamily]{Ligatures=TeX,Scale=1}
\fi
\usepackage{lmodern}
\ifPDFTeX\else
  % xetex/luatex font selection
\fi
% Use upquote if available, for straight quotes in verbatim environments
\IfFileExists{upquote.sty}{\usepackage{upquote}}{}
\IfFileExists{microtype.sty}{% use microtype if available
  \usepackage[]{microtype}
  \UseMicrotypeSet[protrusion]{basicmath} % disable protrusion for tt fonts
}{}
\makeatletter
\@ifundefined{KOMAClassName}{% if non-KOMA class
  \IfFileExists{parskip.sty}{%
    \usepackage{parskip}
  }{% else
    \setlength{\parindent}{0pt}
    \setlength{\parskip}{6pt plus 2pt minus 1pt}}
}{% if KOMA class
  \KOMAoptions{parskip=half}}
\makeatother
\usepackage{xcolor}
\usepackage[margin=1in]{geometry}
\usepackage{color}
\usepackage{fancyvrb}
\newcommand{\VerbBar}{|}
\newcommand{\VERB}{\Verb[commandchars=\\\{\}]}
\DefineVerbatimEnvironment{Highlighting}{Verbatim}{commandchars=\\\{\}}
% Add ',fontsize=\small' for more characters per line
\usepackage{framed}
\definecolor{shadecolor}{RGB}{248,248,248}
\newenvironment{Shaded}{\begin{snugshade}}{\end{snugshade}}
\newcommand{\AlertTok}[1]{\textcolor[rgb]{0.94,0.16,0.16}{#1}}
\newcommand{\AnnotationTok}[1]{\textcolor[rgb]{0.56,0.35,0.01}{\textbf{\textit{#1}}}}
\newcommand{\AttributeTok}[1]{\textcolor[rgb]{0.13,0.29,0.53}{#1}}
\newcommand{\BaseNTok}[1]{\textcolor[rgb]{0.00,0.00,0.81}{#1}}
\newcommand{\BuiltInTok}[1]{#1}
\newcommand{\CharTok}[1]{\textcolor[rgb]{0.31,0.60,0.02}{#1}}
\newcommand{\CommentTok}[1]{\textcolor[rgb]{0.56,0.35,0.01}{\textit{#1}}}
\newcommand{\CommentVarTok}[1]{\textcolor[rgb]{0.56,0.35,0.01}{\textbf{\textit{#1}}}}
\newcommand{\ConstantTok}[1]{\textcolor[rgb]{0.56,0.35,0.01}{#1}}
\newcommand{\ControlFlowTok}[1]{\textcolor[rgb]{0.13,0.29,0.53}{\textbf{#1}}}
\newcommand{\DataTypeTok}[1]{\textcolor[rgb]{0.13,0.29,0.53}{#1}}
\newcommand{\DecValTok}[1]{\textcolor[rgb]{0.00,0.00,0.81}{#1}}
\newcommand{\DocumentationTok}[1]{\textcolor[rgb]{0.56,0.35,0.01}{\textbf{\textit{#1}}}}
\newcommand{\ErrorTok}[1]{\textcolor[rgb]{0.64,0.00,0.00}{\textbf{#1}}}
\newcommand{\ExtensionTok}[1]{#1}
\newcommand{\FloatTok}[1]{\textcolor[rgb]{0.00,0.00,0.81}{#1}}
\newcommand{\FunctionTok}[1]{\textcolor[rgb]{0.13,0.29,0.53}{\textbf{#1}}}
\newcommand{\ImportTok}[1]{#1}
\newcommand{\InformationTok}[1]{\textcolor[rgb]{0.56,0.35,0.01}{\textbf{\textit{#1}}}}
\newcommand{\KeywordTok}[1]{\textcolor[rgb]{0.13,0.29,0.53}{\textbf{#1}}}
\newcommand{\NormalTok}[1]{#1}
\newcommand{\OperatorTok}[1]{\textcolor[rgb]{0.81,0.36,0.00}{\textbf{#1}}}
\newcommand{\OtherTok}[1]{\textcolor[rgb]{0.56,0.35,0.01}{#1}}
\newcommand{\PreprocessorTok}[1]{\textcolor[rgb]{0.56,0.35,0.01}{\textit{#1}}}
\newcommand{\RegionMarkerTok}[1]{#1}
\newcommand{\SpecialCharTok}[1]{\textcolor[rgb]{0.81,0.36,0.00}{\textbf{#1}}}
\newcommand{\SpecialStringTok}[1]{\textcolor[rgb]{0.31,0.60,0.02}{#1}}
\newcommand{\StringTok}[1]{\textcolor[rgb]{0.31,0.60,0.02}{#1}}
\newcommand{\VariableTok}[1]{\textcolor[rgb]{0.00,0.00,0.00}{#1}}
\newcommand{\VerbatimStringTok}[1]{\textcolor[rgb]{0.31,0.60,0.02}{#1}}
\newcommand{\WarningTok}[1]{\textcolor[rgb]{0.56,0.35,0.01}{\textbf{\textit{#1}}}}
\usepackage{graphicx}
\makeatletter
\def\maxwidth{\ifdim\Gin@nat@width>\linewidth\linewidth\else\Gin@nat@width\fi}
\def\maxheight{\ifdim\Gin@nat@height>\textheight\textheight\else\Gin@nat@height\fi}
\makeatother
% Scale images if necessary, so that they will not overflow the page
% margins by default, and it is still possible to overwrite the defaults
% using explicit options in \includegraphics[width, height, ...]{}
\setkeys{Gin}{width=\maxwidth,height=\maxheight,keepaspectratio}
% Set default figure placement to htbp
\makeatletter
\def\fps@figure{htbp}
\makeatother
\setlength{\emergencystretch}{3em} % prevent overfull lines
\providecommand{\tightlist}{%
  \setlength{\itemsep}{0pt}\setlength{\parskip}{0pt}}
\setcounter{secnumdepth}{-\maxdimen} % remove section numbering
\ifLuaTeX
  \usepackage{selnolig}  % disable illegal ligatures
\fi
\IfFileExists{bookmark.sty}{\usepackage{bookmark}}{\usepackage{hyperref}}
\IfFileExists{xurl.sty}{\usepackage{xurl}}{} % add URL line breaks if available
\urlstyle{same}
\hypersetup{
  pdftitle={Almond Yield Assignment 2},
  pdfauthor={Patty Park \& Vanessa Salgado},
  hidelinks,
  pdfcreator={LaTeX via pandoc}}

\title{Almond Yield Assignment 2}
\author{Patty Park \& Vanessa Salgado}
\date{}

\begin{document}
\maketitle

\hypertarget{almond-yield-assignement-2}{%
\section{Almond Yield Assignement 2}\label{almond-yield-assignement-2}}

\hypertarget{draw-diagram-to-represent-your-model---how-it-will-translate-inputs-to-outputs-with-parameters-that-shape}{%
\subsection{1. Draw diagram to represent your model - how it will
translate inputs to outputs, with parameters that
shape}\label{draw-diagram-to-represent-your-model---how-it-will-translate-inputs-to-outputs-with-parameters-that-shape}}

the relationship between inputs an outputs - on your diagram list what
your inputs, parameters and outputs are with units

\begin{itemize}
\tightlist
\item
  Inputs:
\end{itemize}

\begin{enumerate}
\def\labelenumi{\arabic{enumi}.}
\tightlist
\item
  Climate Dataset
\end{enumerate}

\begin{itemize}
\tightlist
\item
  Parametes:
\end{itemize}

\begin{enumerate}
\def\labelenumi{\arabic{enumi}.}
\tightlist
\item
  Precipitation in mm for January
\item
  Minimum temperature in Celcius for February
\end{enumerate}

\begin{itemize}
\tightlist
\item
  Outputs:
\end{itemize}

\begin{enumerate}
\def\labelenumi{\arabic{enumi}.}
\tightlist
\item
  Mean Almond Yield
\item
  Maximum Almond Yield
\item
  Minimum Almond Yield
\end{enumerate}

\hypertarget{diagram-for-our-almond-yield-function}{%
\subsubsection{Diagram for our Almond Yield
Function}\label{diagram-for-our-almond-yield-function}}

{[}INSERT HERE{]}

\hypertarget{implement-your-diagram-as-an-r-function}{%
\subsection{2. Implement your diagram as an R
function}\label{implement-your-diagram-as-an-r-function}}

Our goal is to implement a simple model of almond yield anomaly response
to climate The climate dataframe includes daily time series of minimum,
maximum daily temperatures and precipitation.

We would like to ouput the yields for yearly almond yields.

Our model will use the Almond transfer function stated below :

\[Almond Yield = -0.015T_{n,2} - 0.0046T^2_{n,2} - 0.07P_1 + 0.0043P^2_1 + 0.28\]
\{source: Lobell et al.(2006)\}

where:

\(T_n =\)Minimum Temperature in Celcius for the Month of Janurary

\(P_1 =\)Precipiation in mm for the Month of February

\begin{Shaded}
\begin{Highlighting}[]
\CommentTok{\# {-}{-}{-}{-}{-}{-}{-}{-}{-}{-}{-}{-}{-}{-}{-}{-}{-}{-}{-}{-}{-}{-}{-}{-}{-}{-}{-}{-}{-}{-}{-}{-}{-}{-}{-}{-}{-}{-}}
\CommentTok{\#               Read In Data}
\CommentTok{\# {-}{-}{-}{-}{-}{-}{-}{-}{-}{-}{-}{-}{-}{-}{-}{-}{-}{-}{-}{-}{-}{-}{-}{-}{-}{-}{-}{-}{-}{-}{-}{-}{-}{-}{-}{-}{-}{-}{-}}
\CommentTok{\#read in txt file}
\CommentTok{\#climate\_data \textless{}{-} read.table(file="../Data/clim.txt", sep="")}

\CommentTok{\#write the csv to have for future use}
\CommentTok{\# write\_csv(climate\_data, here::here("Data", "clim.csv"))}

\NormalTok{climate\_data }\OtherTok{\textless{}{-}} \FunctionTok{read\_csv}\NormalTok{(}\FunctionTok{here}\NormalTok{(}\StringTok{"Data"}\NormalTok{, }\StringTok{"clim.csv"}\NormalTok{))}
\end{Highlighting}
\end{Shaded}

\hypertarget{data-cleaning}{%
\subsubsection{Data Cleaning}\label{data-cleaning}}

The climate data is going to need to be preprocessed becase the the
model uses climate data from a particular month, yet we have ALL of the
climate measurements for multiple times per day.

We chose to preprocess within our function to make it easier for the
user to get the necessary output they are interested in. Therefore the
function takes in our climate dataset as an csv as an input.

Below is how we did the data cleaning portion. Note that we used the
minimum temperature found in the daily reading, which is outputted for
the whole year. We also summed the precipitation amount also found in
the daily reading which is outputted for the whole year.

\begin{Shaded}
\begin{Highlighting}[]
\CommentTok{\# {-}{-}{-}{-}{-}{-}{-}{-}{-}{-}{-}{-}{-}{-}{-}{-}{-}{-}{-}{-}{-}{-}{-}{-}{-}{-}{-}{-}{-}{-}{-}{-}{-}{-}{-}{-}{-}{-}{-}{-}{-}{-}{-}{-}{-}{-}{-}}
\CommentTok{\#                  data cleaning }
\CommentTok{\# {-}{-}{-}{-}{-}{-}{-}{-}{-}{-}{-}{-}{-}{-}{-}{-}{-}{-}{-}{-}{-}{-}{-}{-}{-}{-}{-}{-}{-}{-}{-}{-}{-}{-}{-}{-}{-}{-}{-}{-}{-}{-}{-}{-}{-}{-}{-}}
  
\CommentTok{\# subset min\_temp for feb }
\NormalTok{min\_temp }\OtherTok{\textless{}{-}}\NormalTok{ climate\_data }\SpecialCharTok{\%\textgreater{}\%} 
  \FunctionTok{group\_by}\NormalTok{(month, year) }\SpecialCharTok{\%\textgreater{}\%} \CommentTok{\# grouping by month and year}
  \FunctionTok{filter}\NormalTok{(month }\SpecialCharTok{==} \DecValTok{2}\NormalTok{) }\SpecialCharTok{\%\textgreater{}\%} \CommentTok{\# select just feb}
  \FunctionTok{summarise}\NormalTok{(}\AttributeTok{feb\_tmin\_c =} \FunctionTok{mean}\NormalTok{(tmin\_c)) }\SpecialCharTok{\%\textgreater{}\%} \CommentTok{\#find mean temp per year}
  \FunctionTok{group\_by}\NormalTok{() }\SpecialCharTok{\%\textgreater{}\%} 
  \FunctionTok{select}\NormalTok{(}\SpecialCharTok{{-}}\NormalTok{month) }\CommentTok{\#take our month from dataframe}
  
\CommentTok{\# calculate total january precipitation for each year}
\NormalTok{precip }\OtherTok{\textless{}{-}}\NormalTok{ climate\_data }\SpecialCharTok{\%\textgreater{}\%}
  \FunctionTok{group\_by}\NormalTok{(month, year) }\SpecialCharTok{\%\textgreater{}\%}
  \FunctionTok{filter}\NormalTok{(month }\SpecialCharTok{==} \DecValTok{1}\NormalTok{) }\SpecialCharTok{\%\textgreater{}\%} \CommentTok{\# select just jan}
  \FunctionTok{summarise}\NormalTok{(}\AttributeTok{jan\_precip\_mm =} \FunctionTok{sum}\NormalTok{(precip)) }\SpecialCharTok{\%\textgreater{}\%} \CommentTok{\#find sum of precip per year}
  \FunctionTok{group\_by}\NormalTok{() }\SpecialCharTok{\%\textgreater{}\%} 
  \FunctionTok{select}\NormalTok{(}\SpecialCharTok{{-}}\NormalTok{month)}
  
\CommentTok{\# join dataset together}
\NormalTok{yield\_anomaly }\OtherTok{\textless{}{-}} \FunctionTok{left\_join}\NormalTok{(min\_temp, precip) }
\end{Highlighting}
\end{Shaded}

\hypertarget{find-almond-yield}{%
\subsubsection{Find Almond Yield}\label{find-almond-yield}}

After cleaning up the dataset, we are now going to input those value
into our equations to give us the almond yields we are interested in. In
this case, we want to know the maximum, minimum, and mean yields over
all of the years.

\begin{Shaded}
\begin{Highlighting}[]
\CommentTok{\#create the equation needed to find almond yield}
\NormalTok{yield\_anomaly }\OtherTok{\textless{}{-}}\NormalTok{ yield\_anomaly }\SpecialCharTok{\%\textgreater{}\%} 
  \FunctionTok{mutate}\NormalTok{(}\AttributeTok{yield\_anomaly =} \SpecialCharTok{{-}}\FloatTok{0.015}\SpecialCharTok{*}\NormalTok{feb\_tmin\_c }\SpecialCharTok{{-}} \FloatTok{0.0046}\SpecialCharTok{*}\NormalTok{feb\_tmin\_c}\SpecialCharTok{\^{}}\DecValTok{2} \SpecialCharTok{{-}} \FloatTok{0.07}\SpecialCharTok{*}\NormalTok{jan\_precip\_mm }\SpecialCharTok{+} \FloatTok{0.0043}\SpecialCharTok{*}\NormalTok{jan\_precip\_mm}\SpecialCharTok{\^{}}\DecValTok{2} \SpecialCharTok{+} \FloatTok{0.28}\NormalTok{) }\CommentTok{\# calc based on  equation from lobell et al. 2006}

\CommentTok{\# calculate the min, max, and mean yield over the whole time period}
\NormalTok{min\_yield }\OtherTok{\textless{}{-}} \FunctionTok{min}\NormalTok{(yield\_anomaly}\SpecialCharTok{$}\NormalTok{yield\_anomaly)}
\NormalTok{max\_yield }\OtherTok{\textless{}{-}} \FunctionTok{max}\NormalTok{(yield\_anomaly}\SpecialCharTok{$}\NormalTok{yield\_anomaly)}
\NormalTok{mean\_yield }\OtherTok{\textless{}{-}} \FunctionTok{mean}\NormalTok{(yield\_anomaly}\SpecialCharTok{$}\NormalTok{yield\_anomaly)}

\CommentTok{\# print the min, max, and mean}
\FunctionTok{print}\NormalTok{(}\FunctionTok{paste}\NormalTok{(}\StringTok{"Minimum Yield:"}\NormalTok{, }\FunctionTok{round}\NormalTok{(min\_yield, }\DecValTok{2}\NormalTok{), }\StringTok{"ton(s) per acre |"}\NormalTok{,}
                     \StringTok{"Maximum Yield:"}\NormalTok{, }\FunctionTok{round}\NormalTok{(max\_yield, }\DecValTok{2}\NormalTok{), }\StringTok{"ton(s) per acre |"}\NormalTok{,}
                     \StringTok{"Mean Yield:"}\NormalTok{, }\FunctionTok{round}\NormalTok{(mean\_yield, }\DecValTok{2}\NormalTok{), }\StringTok{"ton(s) per acre"}\NormalTok{))}
\end{Highlighting}
\end{Shaded}

\begin{verbatim}
## [1] "Minimum Yield: -0.36 ton(s) per acre | Maximum Yield: 1919.98 ton(s) per acre | Mean Yield: 181.45 ton(s) per acre"
\end{verbatim}

Below is how you can call in the function created on an R script.

\begin{Shaded}
\begin{Highlighting}[]
\CommentTok{\# sourcing in almond\_yield function}
\FunctionTok{source}\NormalTok{(}\StringTok{"almond\_yield.R"}\NormalTok{)}

\FunctionTok{almond\_yield}\NormalTok{(climate\_data)}
\end{Highlighting}
\end{Shaded}

\begin{verbatim}
## [1] "Minimum Yield:-0.36ton(s) per acre |Maximum Yield:1919.98ton(s) per acre |Mean Yield:181.45ton(s) per acre"
\end{verbatim}

\end{document}
